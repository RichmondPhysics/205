\section{Proper Length, Proper Distance, and Invariant Intervals}
%By Matt Trawick.

\makelabheader %(Space for student name, etc., defined in master.tex)

\bigskip

Open the file \filename{proper\_length\_vs\_proper\_distance.nb} in Mathematica. If you get a pop-up error message, you may need to click \button{Enable Dynamic Content}. Type \button{Ctrl-A} to select all lines, and hit \button{Shift-Enter} to execute them.

The graph you see represents a spacetime diagram (or ``Minkowski diagram'') of a long rod flying through the air at high speed. The two purple lines represent the worldlines of the two ends of the pole. There are also flashbulbs at both ends of the pole; the three large dots on the graph (red, green, and black) represent three flashes given off by them (one by the left, two by the right). You can move the slider to see these events from various reference frames.

\begin{enumerate}[wide]
\item By using the slider and looking at the graph, what is the proper length of the rod?
\answerspace{0.7in}

\item What is the proper distance between the red event and the green event?
\answerspace{0.7in}

\item What is the proper distance between the red event and the black event?
\answerspace{0.7in}

\item If two events occur at two ends of an object, is the proper distance between the events always the same as the proper length of the object itself?
\answerspace{0.7in}

\item Is the separation between the red flash and green flash events timelike, spacelike, or lightlike?  Explain.
\answerspace{0.8in}

\item Is the separation between the green flash and black flash events timelike, spacelike, or lightlike?  Explain.
\answerspace{0.8in}

\end{enumerate}
