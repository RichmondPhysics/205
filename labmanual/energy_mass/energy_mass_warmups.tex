\section{Energy and Mass: Warmup Exercises}
%By Matt Trawick.
\label{lab_energy_mass_warmup}
\makelabheader %(Space for student name, etc., defined in master.tex)

\bigskip

\textbf{Introduction}

The \textit{kinetic energy} $K$ of an object is related to its speed by the equation
\begin{equation}
K=\tfrac{1}{2}mv^2,
\end{equation}
where $m$ is the object's mass and $v$ is its velocity.  In this short set of warmup exercises, you will practice using this relationship as well as the idea of \textit{conservation of energy}.

\bigskip

\textbf{Activity 1: A Simple Problem with Work and Energy}
\begin{enumerate}[labparts]
\item Anna (mass $m=50$~kg) is coasting on her bicycle at an initial speed of $v_{\rm i}=10$~m/s.  What is Anna's kinetic energy $K$?  (Remember to include units in your calculation and your answer!)
\answerspace{0.8in}

\item Anna decides she wants to go faster, so she pedals hard for a little while, doing 1100 joules of work---which increases her kinetic energy by 1100 joules.  Now what is her final velocity $v_{\rm f}$?
\answerspace{0.8in}
\end{enumerate}

\textbf{Activity 2: An Exploding Shell}


An explosive shell with mass $m=6$~kg is fired high into the air.  At the top of its trajectory, when it has a speed of 40~m/s, the shell explodes, breaking into two fragments.

\begin{enumerate}[labparts]
\item What is the kinetic energy $K$ of the shell at the top of its trajectory?
\answerspace{0.8in}

\item In breaking the shell into two fragments, the explosion adds a total of 6000 joules of kinetic energy to the system.  Immediately after the explosion, one fragment has a mass of $m_1=4$~kg and a speed $v_1=50$~m/s.  Find the mass $m_2$ and speed $v_2$ of the other fragment.
\answerspace{1.8in}

\item What additional conservation law did you apply in part (b) besides the conservation of energy?
\answerspace{0.8in}
\end{enumerate}

\vspace{\fill}

Note: Mass conservation seems unremarkable to us now, and under most circumstances it would have been intuitive to our ancestors going back thousands of years as well.  (It's related to object permanence, which most of us aquire as infants, and some of us still remember to use from time to time as grown-ups.)  But it's not obvious how mass conservation applies in the case of a burning log, a boiling pot of water, or even a hot air balloon---precursors of which have existed for thousands of years.  In fact, conservation of mass wasn't firmly established until the late 1700s, when it was rigorously confirmed in experiments by French chemist Antoine Lavoisier.

