\section{A Harmonic Oscillator Potential}
%By Matt Trawick

\makelabheader %(Space for student name, etc., defined in master.tex)

\bigskip

So far, all of the potential functions we've studied have been kind of boxy-looking, with discontinuities in them.  Today we'll look at a case in which the potential energy function $U(x)$ is a continuous function.  

\begin{enumerate}[wide]

\item If the potential energy function $U(x)$ is discontinuous, what does that say about the force at the point of the discontinuity?  Is that physically reasonable?
\answerspace{0.6in}

\item Suppose that a particle is being held in place on a spring with some spring constant $\kappa$  (not to be confused with Boltzmann's constant $k$), such that the equilibrium position of the particle is $x=0$.  Sketch graphs below of $F(x)$ and $U(x)$ for the particle, and write equations for them, too.
\begin{center}
%\vspace{-0.1in}
%\includegraphics{potential_intro/activity_1_figs/F_axes.eps}
\begin{lab_axis}[lab_noticks_4quads,
	width={2.5in}, height={1.5in},
	xlabel={$x$},
	ylabel={$F$},
	]
\end{lab_axis}
\hspace{0.2in}
\begin{lab_axis}[lab_noticks_4quads,
	width={2.5in}, height={1.5in},
	xlabel={$x$},
	ylabel={$U$},
	ymin=-0.3,
	]
\end{lab_axis}
\end{center}

\bigskip

\item Think of a real world example when a small particle (say, an atom) is subject to the potential function that can be approximated by the $U(x)$ described above.  (Hint: particles on springs vibrate.  When would an atom basically be held in place, but be free to vibrate back and forth a bit?)
\answerspace{0.6in}

\item Write the time-independent Schr\"odinger equation below.  So far, all of the solutions that we've seen to the 
Schr\"odinger equation in 1-D have the form $\psi(x) = A\cos(kx)$, $\psi(x) = A\sin(kx)$, or $\psi(x) = Ae^{\pm kx}$.  Which (if any) of these functions satisfy the Schr\"odinger equation in this case?  (If you're not sure, plug them in and see.) 
\answerspace{1.8in}

\pagebreak[2]

To give yourself some sense of what $\psi(x)$ looks like for different values of energy $E$, open the following page in a browser:
$$\verb!https://phet.colorado.edu/en/simulation/legacy/bound-states!$$
To run the simulation, click the \textit{play} icon
($\begin{array}{l}\includegraphics[height=3ex]{particle_in_infinite_well/play_icon.pdf}\end{array}$) 
over the image.
%Latex note: the math array above is a quick way to vertically center the inline image.
Once the simulation loads, you will need to make some adjustments to the display:
\begin{itemize}[nosep]
\item At the bottom of the screen, click the \textit{pause} button 
($\begin{array}{l}\includegraphics[height=3ex]{particle_in_infinite_well/pause_icon.pdf}\end{array}$) 
and then click the \textit{restart} button next to it 
($\begin{array}{l}\includegraphics[height=2.5ex]{particle_in_infinite_well/prev_track_icon.pdf}\end{array}$) 
to reset the time to $t=0$.  (In fact, we won't be messing with time evolution at all in this lab.)
\item Under ``Potential Well'' at the top right, select ``Harmonic Oscillator.''
\item Click on ``Configure Potential'' and set the angular frequency $\omega$ to exactly 1.52~fs$^{-1}$
\item If at any time the simulation seems to misbehave or not respond, you can always hit the ``Reset All'' button and start again. 
\end{itemize}

\item Draw sketches below of the wave functions $\psi_n(x)$ and probability densities $\left|\psi_n(x)\right|^2$ for the three wave functions with the lowest energy, and record the energies associated with them.  
\answerspace{1.6in}

\item Are the allowed energy values and the relative spacing between them the \textit{same} as for a particle in an infinite square well, or are they \textit{different}?  Write an expression for the apparent values of energy $E_n$ as a function of $n$ in this case.
\answerspace{1.0in}

\item Look carefully at the wavefunction $\psi(x)$ for $n=14$.  For the region of $x$ where $U(x)<E$, is $\psi(x)$ a strictly sinusoidal function?  (That is, could it be expressed as a single sine or cosine?)  How can you tell whether it is or is not just by looking at the graph on the screen?  
\answerspace{1.0in}

\item Suppose that the energy of the particle were much higher (say, for $n=20,000$), so that the individual lobes of $\psi(x)$ were indistinguishable from each other.  Which would be greater: the probability of finding the particle near the edge of the well, or near the center of the well at $x = 0$?  How do you know?  Is this consistent with what you would expect classically, say, for an actual block of wood on a big steel spring?
\answerspace{0.8in}
\end{enumerate}
